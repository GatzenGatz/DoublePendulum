\documentclass{article}

\usepackage[ngerman]{babel}
\usepackage{inputenc}
\usepackage[T1]{fontenc}

\usepackage[right=2cm, left=2cm, bottom=2cm, top=2cm]{geometry}

\usepackage{amsmath, amssymb, physics}

\author{Gatze :3}
\date{}
\title{Lagrangian and Hamiltonian Formulations for Simulating a Double Compound Pendulum with Uniform Rods}

\begin{document}
\maketitle
\noindent The kinetic and potential energies are given by
\begin{align*}
	T &= \frac{1}{2}m(\dot{x_1^2}+\dot{y_1^2}+\dot{x_2^2}+\dot{y_2^2})+\frac{1}{2}I(\dot{\varphi_1^2}+\dot{\varphi_2^2})\\
	U &= mg(y_1+y_2)
\end{align*}
The center of mass for each pendelum is located at its center. Thus
\begin{align*}
	\dot{x_1^2}+\dot{y_1^2}&=\frac{\dot{\varphi_1^2}l^2}{4}(\sin^2(\varphi_1)+\cos^2(\varphi_1))\\
	&= \frac{\dot{\varphi_1^2}l^2}{4}\\
	\dot{x_2^2}+\dot{y_2^2}&= l^2((\dot{\varphi_1}\cos(\varphi_1)+\frac{1}{2}\dot{\varphi_2}\cos^2(\varphi_2))+(\dot{\varphi_1}\sin(\varphi_1)+\frac{1}{2}\dot{\varphi_2}\sin^2(\varphi_2)))\\
	&= l^2(\dot{\varphi_1^2}+\frac{1}{4}\dot{\varphi_2^2}+\dot{\varphi_1}\dot{\varphi_2}\cos(\varphi_1)\cos(\varphi_2)+\dot{\varphi_1}\dot{\varphi_2}\sin(\varphi_1)\sin(\varphi_2))\\
	&= l^2(\dot{\varphi_1^2}+\frac{1}{4}\dot{\varphi_2^2}+\dot{\varphi_1}\dot{\varphi_2}\cos(\varphi_1-\varphi_2))
\end{align*}
Plugging in the coordinates with respect to the angle as well as the moment of inertia for a thin rod rotated about its axis we get
\begin{align*}
	T &= \frac{m}{2}\left(\frac{\dot{\varphi_1^2}l^2}{4}+l^2(\dot{\varphi_1^2}+\frac{1}{4}\dot{\varphi_2^2}+\dot{\varphi_1}\dot{\varphi_2}\cos(\varphi_1-\varphi_2))\right)+\frac{1}{2}\frac{1}{12}ml^2(\dot{\varphi_1^2}+\dot{\varphi_2^2})\\
	U &= -mgl(\frac{1}{2}\cos(\varphi_1)+\cos(\varphi_1)+\frac{1}{2}\cos(\varphi_2))
\end{align*}
The Lagrangian $L$ is given by
\begin{align*}
	L &= T-U\\
	&= \frac{ml^2}{2}\left(\frac{\dot{\varphi_1^2}}{4}+\dot{\varphi_1^2}+\frac{1}{4}\dot{\varphi_2^2}+\dot{\varphi_1}\dot{\varphi_2}\cos(\varphi_1-\varphi_2)\right)+\frac{1}{24}ml^2(\dot{\varphi_1^2}+\dot{\varphi_2^2}) + \frac{mgl}{2}(3\cos(\varphi_1)+\cos(\varphi_2))\\
	&= \frac{ml^2}{24}\left(15\dot{\varphi_1^2}+3\dot{\varphi_2^2}+12\dot{\varphi_1}\dot{\varphi_2}\cos(\varphi_1-\varphi_2)\right)+\frac{1}{24}ml^2(\dot{\varphi_1^2}+\dot{\varphi_2^2}) + \frac{mgl}{2}(3\cos(\varphi_1)+\cos(\varphi_2))\\
	&= \frac{ml^2}{6}\left(4\dot{\varphi_1^2}+\dot{\varphi_2^2}+3\dot{\varphi_1}\dot{\varphi_2}\cos(\varphi_1-\varphi_2)\right) + \frac{mgl}{2}(3\cos(\varphi_1)+\cos(\varphi_2))\\
\end{align*}
Using the Euler-Lagrange equations we could solve for the Equations of Motion in the form of Lagrangian mechanics. But these we yield a second-order differential equation. Solving for the Equations of Motion in Hamilton mechanics is better for simulating, because you only get first-order differential equations, which can more be solved more accurately numerically. Therefore we perform a Legendre-Transform on the Lagrangian, which turns it into the Hamiltonian.\\
With the generalized momenta $p=\pdv{L}{\dot{\varphi}}$ we obtain
\begin{align*}
	H &= p\dot{\varphi} - L
\end{align*}
where $p$ and $\dot{\varphi}$ are 2-dimensional vectors. First we solve for the generalized momenta.
\begin{align*}
	p_1 &= \pdv{L}{\dot{\varphi_1}}\\
	&= \frac{8ml^2}{6}\dot{\varphi_1}+\frac{3}{6}\dot{\varphi_2}\cos(\varphi_1-\varphi_2)\\
	p_2 &= \pdv{L}{\dot{\varphi_2}}\\
	&=  \frac{2ml^2}{6}\dot{\varphi_2}+\frac{3}{6}\dot{\varphi_1}\cos(\varphi_1-\varphi_2) 
\end{align*}
This can be written as
\begin{align*}
	a
\end{align*}
Because $H=T+U$ and L=$T-U$, $T=\frac{1}{2}p\dot{\varphi}$. 

\end{document}
